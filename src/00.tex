\maketitle
\section*{Motivation: Why covering spaces?}
\todo[inline]{Topologists have a love-hate relationship with topological spaces.}
One way to understand spaces is to associate an algebraic invariant to them.
\begin{equation*}
  \begin{tikzcd}
    \mbox{Topological Space} \ar[rr] && \mbox{Algebraic invariant} \\
    \mbox{For example: Knot} \ar[r] & \mbox{Knot projection} \ar[r] & \mbox{Knot invariant}
  \end{tikzcd}
\end{equation*}
where a knot invariant can be a number, polynomial, vector space, chain complex, etc.

It is rarely possible to compute an invariant directly using definition.
Instead, there are two main strategies in algebraic topology to compute these algebraic invariants:
\begin{enumerate}
  \item Break a space $X$ up into smaller spaces,
  \begin{align*}
    X = X_1 \cup \dots \cup X_n.
  \end{align*}
  This is called \emph{excision}.
  \item Map another space $Y$ into $X$ and recover information about $X$ using $Y$.
  \begin{align*}
    Y \longrightarrow X
  \end{align*}
  This is where covering spaces come in. A covering map $Y \rightarrow X$ provides us information about the fundamental group/first homotopy group $\pi_1(X)$.
  We will show later on that if $\pi_1(Y) \triangleleft \pi_1(X)$ then there is a short exact sequence of groups
  \begin{equation*}
    1 \rightarrow \pi_1(Y) \rightarrow \pi_1(X) \rightarrow \Gal(Y|X) \rightarrow 1
  \end{equation*}
\end{enumerate}
